\documentclass[
12pt, % 字体大小
a4paper, 
oneside, % 单面打印(双面为twoside)
headinclude,footinclude, % 页眉页脚包含在文本区域内,确保不被裁剪或掩盖
]{scrartcl}
% 主题和样式
\usepackage[
nochapters, % 无章节层级 
beramono, % 等宽字体样式
eulermath, % 数学公式Euler字体
pdfspacing, % 字间距
dottedtoc % 点线式目录
]{classicthesis}
\usepackage{arsclassica} 
%----------------------------------------------------------------------------------------
% 输入和页面排版
\usepackage[T1]{fontenc} % 字体编码
\usepackage[utf8]{inputenc} % 输入编码
\usepackage{ctex} % 汉语
\usepackage{amsmath,amssymb,amsthm} % 数学公式
\usepackage{indentfirst} % 缩进
\setlength{\parindent}{2em} % 段落缩进
\usepackage[
top=2cm,
bottom=2cm, 
left=2cm,
right=2cm, 
headheight=20pt, 
includeheadfoot 
]{geometry} % 页面
\usepackage{scrlayer-scrpage} % 页眉页脚
\renewcommand{\sectionmark}[1]{\markright{\spacedlowsmallcaps{#1}}}
\renewcommand{\subsectionmark}[1]{\markright{\thesubsection~#1}}
\lehead{\mbox{\llap{\small\thepage\kern1em\color{halfgray} \vline}\color{halfgray}\hspace{0.5em}\rightmark\hfil}} % 标题旁边标记页码
\cfoot{\hyperlink{toc}{\color{RoyalBlue}返回目录}} % 页脚返回目录链接
\pagestyle{scrheadings}
%----------------------------------------------------------------------------------------
% 图表和引用
\usepackage{graphicx} % 图像
\graphicspath{{Figures/}} % 图像路径
\usepackage{subfig} % 图组
\usepackage{float} % 浮动
\usepackage{enumitem} % 列表
\usepackage{varioref} % 交叉引用
%----------------------------------------------------------------------------------------
% 代码
\usepackage{listings}
\lstset{
    language=Matlab,
    basicstyle=\ttfamily\small,   % 字体
    numbers=left,                 % 行号
    numberstyle=\tiny\color{gray},
    stepnumber=5,
    numbersep=5pt,
    backgroundcolor=\color{white},% 背景
    tabsize=2,                    % 制表符宽度
    frame=single,                 % 边框
    captionpos=t,                 % 标题
    title=\lstname,
    breaklines=true,              % 换行
    breakatwhitespace=true,
    escapeinside={`}{`},          % 转义(中文注释)
}
\lstset{
    language=Python,            
    basicstyle=\ttfamily\small,   % 字体
    numbers=left,                 % 行号
    numberstyle=\tiny\color{gray}, 
    stepnumber=5,             
    numbersep=5pt,            
    backgroundcolor=\color{white},% 背景
    tabsize=4,                    % 制表符宽度            
    frame=single,                 % 边框
    captionpos=t,                 % 标题
    title=\lstname, 
    breaklines=true,              % 换行
    breakatwhitespace=false,   
    escapeinside={`}{`},          % 转义(中文注释)
}
\usepackage{algorithm} % 算法
\usepackage{algpseudocode}
\usepackage{mdframed} % 跨页框架
% 不浮动算法环境
\newcounter{myalgorithm}
\renewcommand{\themyalgorithm}{\arabic{myalgorithm}}
\newenvironment{myalgorithm}[1][]{
  \refstepcounter{myalgorithm}
  \begin{mdframed}[
    skipabove=\topskip,
    skipbelow=\topskip,
    needspace=3\baselineskip,
    linewidth=0.4pt,
    frametitlefont=\normalfont\bfseries,
    frametitle={算法 \themyalgorithm\if\relax\detokenize{#1}\relax\else:#1\fi},
    frametitlerule=true,
    frametitlerulewidth=0.4pt,
    repeatframetitle=true
  ]
  \begin{algorithmic}[1]
  \ifx\relax\detokenize{#1}\relax
    \addcontentsline{alg}{algorithms}{\makebox[7em][l]{算法~\themyalgorithm} }
  \else
    \addcontentsline{alg}{algorithms}{\makebox[7em][l]{算法~\themyalgorithm} #1}
  \fi
}{
  \end{algorithmic}
  \end{mdframed}
}
% 关键词
\algrenewcommand{\algorithmicwhile}{当}
\algrenewcommand{\algorithmicdo}{执行}
\algrenewcommand{\algorithmicend}{结束}
\algrenewcommand{\algorithmicif}{如果}
\algrenewcommand{\algorithmicthen}{那么}
\algrenewcommand{\algorithmicelse}{否则}
\algrenewcommand{\algorithmicfor}{对于}
\algrenewcommand{\algorithmicrepeat}{循环}
\algrenewcommand{\algorithmicuntil}{直到}
\algrenewcommand{\algorithmicloop}{循环}
\algnotext{EndFor}
\algnotext{EndIf}
\algnotext{EndLoop}
\algnotext{EndWhile}
%----------------------------------------------------------------------------------------
% 超链接与PDF信息
\usepackage{hyperref} 
\hypersetup{
colorlinks=true, % 彩色
breaklinks=true, % 断行
urlcolor=webbrown, % URL棕色
linkcolor=RoyalBlue, % 内部链接蓝色
citecolor=webgreen, % 引用绿色
bookmarks=true, % 书签
bookmarksnumbered,
pdftitle={}, 
pdfauthor={},
pdfsubject={}, 
pdfkeywords={}, 
pdfcreator={pdfLaTeX}, 
pdfproducer={LaTeX with hyperref and ClassicThesis} 
}
%----------------------------------------------------------------------------------------
% 目录与标题
\usepackage{titlesec} 
\AtBeginDocument{
    \renewcommand{\contentsname}{目\hspace{1em}录}
    \renewcommand{\listfigurename}{图\hspace{1em}片}
    \renewcommand{\listtablename}{表\hspace{1em}格}
    \renewcommand{\figurename}{图}
    \renewcommand{\tablename}{表}
    \setcounter{tocdepth}{3} % 目录深度
}
\theoremstyle{definition} 
\newtheorem{definition}{定义}
\theoremstyle{plain} 
\newtheorem{theorem}{定理}
\theoremstyle{remark}
\newtheorem{remark}{备注}
\newtheorem{example}{样例}
\usepackage{tocloft} % 目录
% 要点目录
\newlistof{tips}{tip}{要\hspace{1em}点}
\newcommand{\tip}[1]{
  \refstepcounter{tips}
  \textsuperscript{\textcolor{orange}{\textbf{\thetips}}}
  \addcontentsline{tip}{tips}{\makebox[7em][l]{要点~\thetips} #1}
}
% 算法目录
\newlistof{algorithms}{alg}{算\hspace{1em}法} 
\hyphenation{Fortran hy-phen-ation} % 单词断字规则
%----------------------------------------------------------------------------------------
% 题目和作者
\title{\normalfont\spacedallcaps{深度学习}} 
\date{}
%----------------------------------------------------------------------------------------
% 开始和目录
\begin{document}
\maketitle
\newpage
\hypertarget{toc}{}
\begingroup
\begin{multicols}{2}
\tableofcontents
\end{multicols}
\endgroup
\newpage
\listoffigures
\listoftables
\listoftips
\newpage
%----------------------------------------------------------------------------------------
\section{人工智能学家}
\begin{table}[h]
    \centering
    \begin{tabular}{|l|c|c|}
        \hline
        姓名 & 图灵奖 & 诺贝尔奖 \\
        \hline
        Geoffrey Hinton & 1 & 1 \\
        \hline
        John Hopfield & 0 & 1 \\
        \hline
    \end{tabular}
\end{table}

\section{深度学习"深度":神经网络的深度(层数)}

\section{前馈神经网络 (Feed-Forward Neural Network, FNN)}
特点:相邻层间特征是单向连接。

\section{全连接神经网络 (Full Connect Neural Network, FCNN)}
\begin{itemize}
    \item 前向传播: 计算结果并保存特征。
    \item 反向传播: 链式规则。
\end{itemize}

\section{卷积神经网络 (Convolutional Neural Networks, CNN)}
\begin{itemize}
    \item 结构
    \begin{itemize}
        \item 卷积层: 提取局部特征。
        \item 池化层: 降低特征维度。
        \item 全连接层: 分类。
    \end{itemize}
    
    \item 卷积计算过程\\
    原图像大小: $H_0 \times N_0 \times M_0 \times A_0$\\
    卷积核大小: B × F × F × $A_0$\\
    Padding: P\\
    Stride: S\\
    新图像大小: $N_1 = \frac{N_0+2P-F}{S}+1, A_1 = B$\\
    参数量: $F \times F \times A_0 \times B$
    
    \item 池化计算过程\\
    最大池化 (Max Pooling):选择区域的最大值作为代表性的特征值。\\
    平均池化 (Average Pooling):计算区域的平均值作为代表性的特征值。\\
    池化层参数量为 0。
    
    \item 填充 (Padding)\\
    增加感受野,减少信息损失:确保边缘像素能被卷积核充分覆盖,得到有效处理,而不是丢失。\\
    控制输出尺寸:通过调整填充量可以精确控制每一层的输出尺寸。
    
    \item 步幅 (Stride):\\
    控制输出尺寸、下采样程度:较大步幅可以减小输出的空间尺寸,降低计算复杂度,减少参数量,提高特征图的缩放比例。\\
    调整感受野:较大步幅意味着输出单元会覆盖较大输入区域,增加感受野,减少重叠区域数量。\\
    平衡速度与精度:较大步幅可以加速计算过程,但可能丢失细节信息;较小步幅能更精细地捕捉特征,但会增加计算成本。
    
    \item 1×1 卷积\\
    维度变换(降维/升维):改变特征图的深度(通道数),降维有助于降低模型复杂度和计算量,同时保持大部分有用信息。\\
    非线性引入:在卷积后添加激活函数,可以在不改变空间尺寸的情况下引入非线性,使模型能够学习更复杂的模式。\\
    作为瓶颈层:在一些架构中(如 ResNet、Inception),可以用作瓶颈层(先 1×1 卷积降维,再进行其他卷积,最后 1×1 卷积恢复维度)。显著减少参数数量和计算成本,同时维持性能。\\
    特征融合:融合不同尺度或不同来源的特征图,合并成一个新的特征表示。
\end{itemize}



\begin{itemize}
    \item 实例
    \begin{itemize}
        \item LeNet: 没有使用ReLU。
        \item AlexNet: 最早使用了ReLU、GPU。
        \item VGGNet: 小卷积核 (感受野上, 3个3×3=1个7×7)。
        \item GoogleNet: 使用了ReLU, Inception。1×1 卷积
        \item ResNet: 使用了ReLU, 恒等映射直连边,残差模块。
        \item 趋势:卷积核变小、层数增加,抛弃池化层、全连接层。
    \end{itemize}
\end{itemize}

\section{损失函数}
交叉熵损失函数更适用于分类问题,常用于衡量模型预测的概率分布与真实标记
的概率分布之间的差异。
%----------------------------------------------------------------------------------------
\end{document}