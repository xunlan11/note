% 主题和样式
\usepackage[
nochapters, % 无章节层级 
beramono, % 等宽字体
eulermath, % 数学公式Euler字体
pdfspacing, % 字间距
dottedtoc % 点线式目录
]{classicthesis}
\usepackage{arsclassica} 
%----------------------------------------------------------------------------------------
% 输入和页面
\usepackage[T1]{fontenc} % 字体编码
\usepackage[utf8]{inputenc} % 输入编码
\usepackage{ctex} % 汉语
\usepackage{amsmath,amssymb,amsthm} % 数学公式
\usepackage{indentfirst} % 缩进
\setlength{\parindent}{2em} % 段落缩进
\usepackage[
top=2cm,
bottom=2cm, 
left=2cm,
right=2cm, 
headheight=20pt, 
includeheadfoot 
]{geometry} % 页面
\usepackage{scrlayer-scrpage} % 页眉页脚
\renewcommand{\sectionmark}[1]{\markright{\spacedlowsmallcaps{#1}}}  % 获取章节标题
\renewcommand{\subsectionmark}[1]{\markright{\thesubsection~#1}}
\lehead{\mbox{\llap{\small\thepage\kern1em\color{halfgray} \vline}\color{halfgray}\hspace{0.5em}\rightmark\hfil}} % 标题旁边标记页码
\cfoot{\hyperlink{toc}{\color{RoyalBlue}返回目录}} % 页脚返回目录链接
\pagestyle{scrheadings}
% 颜色
\usepackage{xcolor}
\definecolor{mybg}{RGB}{250,250,245}
\definecolor{mygray}{RGB}{120,120,120}
\definecolor{mygreen}{RGB}{34,139,34}
\definecolor{myblue}{RGB}{0,102,204}
\definecolor{myred}{RGB}{200,0,0}
\definecolor{myorange}{RGB}{255,140,0}
\definecolor{mypurple}{RGB}{153,51,255}
%----------------------------------------------------------------------------------------
% 图表和引用
\usepackage{graphicx} % 图像
\graphicspath{{Figures/}} % 图像路径
\usepackage{subfig} % 图组
\usepackage{float} % 浮动
\usepackage{enumitem} % 列表
\usepackage{multirow,multicol} % 多行多列
\usepackage{varioref} % 交叉引用
\usepackage{tikz} % 绘图
\usepackage{pgfplots}
\pgfplotsset{compat=1.18}
%----------------------------------------------------------------------------------------
% 代码
\usepackage{listings}
\lstset{
    language=C++,                     
    basicstyle=\ttfamily\small,    % 字体
    numbers=left,                  % 行号
    numberstyle=\tiny\color{mygray},  
    stepnumber=5,
    numbersep=5pt,
    backgroundcolor=\color{mybg},  % 背景
    showspaces=false,              % 空格与制表符
    showstringspaces=false,
    tabsize=4,                    
    frame=single,                  % 边框
    captionpos=t,                  % 标题
    title=\lstname,
    breaklines=true,               % 换行
    breakatwhitespace=true,
    escapeinside={`}{`},           % 转义(中文注释)
}
\lstset{
    language=Python,            
    basicstyle=\ttfamily\small,    % 字体
    numbers=left,                  % 行号
    numberstyle=\tiny\color{mygray}, 
    stepnumber=5,             
    numbersep=5pt,            
    backgroundcolor=\color{mybg},  % 背景
    showspaces=false,              % 空格与制表符
    showstringspaces=false,
    tabsize=4,                     
    frame=single,                  % 边框
    captionpos=t,                  % 标题
    title=\lstname, 
    breaklines=true,               % 换行
    breakatwhitespace=false,   
    escapeinside={`}{`},           % 转义(中文注释)
}
\lstset{
    language=Matlab,
    basicstyle=\ttfamily\small,    % 字体
    numbers=left,                  % 行号
    numberstyle=\tiny\color{mygray},
    stepnumber=5,
    numbersep=5pt,
    backgroundcolor=\color{mybg},  % 背景
    showspaces=false,              % 空格与制表符
    showstringspaces=false,
    tabsize=2,                     
    frame=single,                  % 边框
    captionpos=t,                  % 标题
    title=\lstname,
    breaklines=true,               % 换行
    breakatwhitespace=true,
    escapeinside={`}{`},           % 转义(中文注释)
}
\lstset{
    language=bash,
    basicstyle=\ttfamily\small,    % 字体
    numbers=left,                  % 行号
    numberstyle=\tiny\color{mygray},
    stepnumber=5,
    numbersep=5pt,
    backgroundcolor=\color{mybg},  % 背景
    showspaces=false,              % 空格与制表符
    showstringspaces=false,
    tabsize=4,                                
    frame=single,                  % 边框
    captionpos=t,                  % 标题
    title=\lstname,
    breaklines=true,               % 换行
    breakatwhitespace=true,
    escapeinside={`}{`},           % 转义(中文注释)
}
\usepackage{algorithm} % 算法
\usepackage{algpseudocode}
\usepackage{mdframed} % 跨页框架
% 不浮动算法环境
\newcounter{myalgorithm}
\renewcommand{\themyalgorithm}{\arabic{myalgorithm}}
\newenvironment{myalgorithm}[1][]{
  \refstepcounter{myalgorithm}
  \begin{mdframed}[
    skipabove=\topskip,
    skipbelow=\topskip,
    needspace=3\baselineskip,
    linewidth=0.4pt,
    frametitlefont=\normalfont\bfseries,
    frametitle={算法 \themyalgorithm\if\relax\detokenize{#1}\relax\else:#1\fi},
    frametitlerule=true,
    frametitlerulewidth=0.4pt,
    repeatframetitle=true
  ]
  \begin{algorithmic}[1]
    \addcontentsline{alg}{algorithms}{\makebox[4em][l]{算法~\themyalgorithm} #1}
}{
  \end{algorithmic}
  \end{mdframed}
}
% 关键词
\algrenewcommand{\algorithmicwhile}{当}
\algrenewcommand{\algorithmicdo}{执行}
\algrenewcommand{\algorithmicend}{结束}
\algrenewcommand{\algorithmicif}{如果}
\algrenewcommand{\algorithmicthen}{那么}
\algrenewcommand{\algorithmicelse}{否则}
\algrenewcommand{\algorithmicfor}{对于}
\algrenewcommand{\algorithmicrepeat}{循环}
\algrenewcommand{\algorithmicuntil}{直到}
\algrenewcommand{\algorithmicloop}{循环}
\algnotext{EndFor}
\algnotext{EndIf}
\algnotext{EndLoop}
\algnotext{EndWhile}
%----------------------------------------------------------------------------------------
% 超链接与PDF信息
\usepackage{hyperref} 
\hypersetup{
colorlinks=true, % 彩色
breaklinks=true, % 断行
urlcolor=webbrown, % URL棕色
linkcolor=RoyalBlue, % 内部链接蓝色
citecolor=webgreen, % 引用绿色
bookmarks=true, % 书签
bookmarksnumbered,
pdftitle={}, 
pdfauthor={},
pdfsubject={}, 
pdfkeywords={}, 
pdfcreator={pdfLaTeX}, 
pdfproducer={LaTeX with hyperref and ClassicThesis} 
}
%----------------------------------------------------------------------------------------
% 目录与标题
\usepackage{tocloft}
\usepackage{titlesec} 
\AtBeginDocument{
    \renewcommand{\contentsname}{目\hspace{1em}录}
    \renewcommand{\listfigurename}{图\hspace{1em}片}
    \setlength{\cftfignumwidth}{4em}
    \renewcommand{\listtablename}{表\hspace{1em}格}
    \setlength{\cfttabnumwidth}{4em}
    \renewcommand{\figurename}{图}
    \renewcommand{\tablename}{表}
    \setcounter{tocdepth}{3} % 目录深度
}
% \subsection[a]{a\tip{a}}
\let\oldsubsection\subsection
\renewcommand{\subsection}[2][]{
  \oldsubsection[#1]{#2}
}
\let\oldsubsubsection\subsubsection
\renewcommand{\subsubsection}[2][]{
  \oldsubsubsection[#1]{#2}
}
% 要点
\newlistof{tips}{tip}{要\hspace{1em}点}
\newcommand{\tip}[1]{
  \refstepcounter{tips}
  \textsuperscript{\textcolor{orange}{\textbf{\thetips}}}
  \addcontentsline{tip}{tips}{\makebox[4em][l]{要点~\thetips} #1}
}
\newlistof{algorithms}{alg}{算\hspace{1em}法} % 算法